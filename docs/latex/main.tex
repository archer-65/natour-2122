\documentclass{natourDoc}
\usepackage{lipsum}
\usepackage{tabularx}
\usepackage[table]{xcolor}
\title{Documentazione IngSW} %Titolo

\begin{document}

%----------- Informazioni --------------------

\titolo{NaTour21} %Titolo.pdf
\sottotitolo{Dipartimento di Ingegneria Elettrica \\ e delle Tecnologie dell'Informazione} %Nome progetto

\professors{Sergio \textsc{Di Martino} \\
Francesco \textsc{Cutugno}} %Nome docenti

\students{Bianca Giada \textsc{Chehade} N86003209 \\
Mario \textsc{Liguori} N86003258\\ 
Mattia \textsc{Rossi} N86003211} %Nome studenti

%----------- Inizializzazione -----------------

\marginscreate %Crea margini
\cover %Crea cover
\toc %Crea Table of Contents

%------------ Corpo del documento ----------------

% \section{Première section} 

% \lipsum[3-4]%LOREMIPSUM

% \subsection{Subsection}

% \lipsum[3-4] %LOREMIPSUM

% \section{Deuxième section}

% \lipsum[3-5] %LOREMIPSUM

%------------- Comandi utili ----------------

% \section{Quelques commandes}

% Voici quelques commandes utiles :

%------ Per inserire immagini centrate ------

%\insertfigure{logos/logo.png}{3cm}{Légende de la figure}{Label de la figure}
% Le premier argument est le chemin pour la photo
% Le deuxième est la hauteur de la photo
% Le troisième la légende
% Le quatrième le label
%Ici, je cite l'image \ref{fig: Label de la figure}


%------- Per inserire un equazione --------------

% \begin{equation} \label{eq: exemple}
	% \rho + \Delta = 42
	% \end{equation}
	
	% L'équation \ref{eq: exemple} est cité ici. 
	
	% ------- Per inserire simboli/variabili ----------------------
	
	% Pour écrire des variables dans le texte, il suffit de mettre le symbole \$ entre le texte souhaité comme : constante $\rho$. 
	
	\section{Descrizione del Progetto}
	NaTour21 è un sistema complesso e distribuito finalizzato ad offrire un moderno social network multipiattaforma per appassionati di escursioni.\\
	
	Il sistema consiste in:
	\begin{itemize}
		\item un Back-End sicuro, performante e scalabile;
		\item un client mobile attraverso cui gli
		utenti possono fruire delle funzionalità del sistema in modo intuitivo, rapido e piacevole;
		\item un client mobile attraverso cui gli amministratori possono gestire i contenuti inseriti in piattaforma.\\
	\end{itemize}
	
	NaTour21 ha lo scopo di creare una community sicura e affidabile dove condividere la propria passione per l'escursionismo.\\
	
	In questo scenario, l'utente si configura come protagonista: oltre alla possibilità di inserire itinerari (dettagliati da informazioni), compilation e post personali sulla piattaforma,
	è lasciato ampio spazio all'individualità personale.\\
	
	Tutto ciò si concretizza con la possibilità di interagire con gli altri utenti, in modo da poter avere un contatto più diretto con
	la realtà dell'escursionismo, e di lasciare valutazioni personali su qualunque itinerario si desideri.\\
	
	Il sistema valuta essenziale la sicurezza degli utenti: questi potranno segnalare informazioni inesatte o contenuti inappropriati, al fine 
	di rendere la permanenza nella community piacevole.
	
	\section{Documento dei Requisiti Software}
	\subsection{Requisiti funzionali}
	In questa sezione saranno esposti i requisiti funzionali dell'applicazione NaTour21, cioè le funzionalità richieste dai commissionanti.
	
	\begin{table}[H]
		\centering
		\begin{tabular}{ |p{5cm}|p{10.3cm}| } 
			\hline
			\rowcolor{PineGreen!70}
			\textbf{Nome} & \textbf{Descrizione} \\
			\hline
			Registrazione & Il sistema deve consentire ad un utente di potersi registrare indicando e-mail e password, oppure utilizzando account di terze parti (Google o Facebook).\\ 
			\hline
		\end{tabular}
		\caption{RQF.1}
		\label{table:1}
	\end{table}
	
	\begin{table}[H]
		\centering
		\begin{tabular}{ |p{5cm}|p{10.3cm}| } 
			\hline
			\rowcolor{PineGreen!70}
			\textbf{Nome} & \textbf{Descrizione} \\
			\hline
			Accesso & Il sistema deve consentire ad un utente di poter effettuare l'accesso alla piattaforma.\\ 
			\hline
		\end{tabular}
		\caption{RQF.2}
		\label{table:2}
	\end{table}
	
	\begin{table}[H]
		\centering
		\begin{tabular}{ |p{5cm}|p{10.3cm}| }
			\hline
			\rowcolor{PineGreen!70}
			\textbf{Nome} & \textbf{Descrizione} \\
			\hline
			Visualizzazione itinerari & Il sistema deve consentire a un utente autenticato di visualizzare i
			dettagli degli itinerari pubblicati e i post ad essi associati. \\
			\hline
		\end{tabular}
		\caption{RQF.3}
		\label{table:3}
	\end{table}

	\begin{table}[H]
		\centering
		\begin{tabular}{ |p{5cm}|p{10.3cm}| } 
			\hline
			\rowcolor{PineGreen!70}
			\textbf{Nome} & \textbf{Descrizione} \\
			\hline
			Inserimento itinerario &  Il sistema deve consentire a un utente autenticato di inserire nuovi itinerari (sentieri) in piattaforma. Un sentiero è
			caratterizzato da un nome, una durata, un livello di difficoltà, un punto di inizio, delle foto (max 5), una descrizione
			(opzionale), e un tracciato geografico (opzionale) che lo rappresenta su una mappa. Il tracciato
			geografico può essere inseribile manualmente (interagendo con una mappa interattiva) oppure
			tramite file in formato standard GPX.\\ 
			\hline
		\end{tabular}
		\caption{RQF.4}
		\label{table:4}
	\end{table}
	
	\begin{table}[H]
		\centering
		\begin{tabular}{ |p{5cm}|p{10.3cm}| } 
			\hline
			\rowcolor{PineGreen!70}
			\textbf{Nome} & \textbf{Descrizione} \\
			\hline
			Ricerca itinerari &  Il sistema deve consentire a un utente autenticato di effettuare ricerche di itinerari tra quelli presenti in piattaforma, con possibilità di filtrare i risultati
			per area geografica, per livello di difficoltà, per durata, e per accessibilità a disabili.\\ 
			\hline
		\end{tabular}
		\caption{RQF.5}
		\label{table:5}
	\end{table}
	
	\begin{table}[H]
		\centering
		\begin{tabular}{ |p{5cm}|p{10.3cm}| } 
			\hline
			\rowcolor{PineGreen!70}
			\textbf{Nome} & \textbf{Descrizione} \\
			\hline
			Valutare itinerari & Il sistema deve consentire a un utente autenticato di indicare un punteggio di difficoltà e/o un tempo
			di percorrenza diverso da quello indicato dall’utente che ha inserito il sentiero. In questo caso, il
			punteggio di difficoltà e il tempo di percorrenza per il sentiero saranno ri-calcolati come la media
			delle difficoltà/dei tempi indicati.\\ 
			\hline
		\end{tabular}
		\caption{RQF.6}
		\label{table:6}
	\end{table}
	
	\begin{table}[H]
		\centering
		\begin{tabular}{ |p{5cm}|p{10.3cm}| }
			\hline
			\rowcolor{PineGreen!70}
			\textbf{Nome} & \textbf{Descrizione} \\
			\hline
			Salvataggio itinerario & Il sistema deve consentire a un utente autenticato di salvare gli itinerari
			nelle proprie compilation personalizzate. \\
			\hline
		\end{tabular}
		\caption{RQF.7}
		\label{table:7}
	\end{table}

	\begin{table}[H]
		\centering
		\begin{tabular}{ |p{5cm}|p{10.3cm}| }
			\hline
			\rowcolor{PineGreen!70}
			\textbf{Nome} & \textbf{Descrizione} \\
			\hline
			Indicazioni itinerario & Il sistema deve consentire a un utente autenticato di ottenere indicazioni
			per un tracciato geografico (se presente). \\
			\hline
		\end{tabular}
		\caption{RQF.8}
		\label{table:8}
	\end{table}

	\begin{table}[H]
		\centering
		\begin{tabular}{ |p{5cm}|p{10.3cm}| }
			\hline
			\rowcolor{PineGreen!70}
			\textbf{Nome} & \textbf{Descrizione} \\
			\hline
			Segnalazione itinerari & Il sistema deve consentire a un utente autenticato di segnalare degli itinerari
			le quali informazioni potrebbero non essere corrette e/o aggiornate. \\
			\hline
		\end{tabular}
		\caption{RQF.9}
		\label{table:9}
	\end{table}

	\begin{table}[H]
		\centering
		\begin{tabular}{ |p{5cm}|p{10.3cm}| }
			\hline
			\rowcolor{PineGreen!70}
			\textbf{Nome} & \textbf{Descrizione} \\
			\hline
			Elimina itinerario & Il sistema deve consentire a un utente autenticato di eliminare itinerari di cui
			è l'autore. \\
			\hline
		\end{tabular}
		\caption{RQF.10}
		\label{table:10}
	\end{table}
	
	\begin{table}[H]
		\centering
		\begin{tabular}{ |p{5cm}|p{10.3cm}| }
			\hline
			\rowcolor{PineGreen!70}
			\textbf{Nome} & \textbf{Descrizione} \\
			\hline
			Visualizzazione post & Il sistema deve consentire a un utente autenticato di visualizzare
			i dettagli di un post. \\
			\hline
		\end{tabular}
		\caption{RQF.11}
		\label{table:11}
	\end{table}

	\begin{table}[H]
		\centering
		\begin{tabular}{ |p{5cm}|p{10.3cm}| } 
			\hline
			\rowcolor{PineGreen!70}
			\textbf{Nome} & \textbf{Descrizione} \\
			\hline
			Inserimento post &  Il sistema deve consentire a un utente autenticato di inserire un post, 
			caratterizzato da foto (max 5), una descrizione (opzionale) e un itinerario associato. \\
			\hline
		\end{tabular}
		\caption{RQF.12}
		\label{table:12}
	\end{table}
	
	
	\begin{table}[H]
		\centering
		\begin{tabular}{ |p{5cm}|p{10.3cm}| }
			\hline
			\rowcolor{PineGreen!70}
			\textbf{Nome} & \textbf{Descrizione} \\
			\hline
			Segnalazione post & Il sistema deve consentire a un utente autenticato di segnalare post con fotografie
			inappropriate. \\
			\hline
		\end{tabular}
		\caption{RQF.13}
		\label{table:13}
	\end{table}

	\begin{table}[H]
		\centering
		\begin{tabular}{ |p{5cm}|p{10.3cm}| }
			\hline
			\rowcolor{PineGreen!70}
			\textbf{Nome} & \textbf{Descrizione} \\
			\hline
			Elimina post & Il sistema deve consentire a un utente autenticato di eliminare post di cui
			è l'autore \\
			\hline
		\end{tabular}
		\caption{RQF.14}
		\label{table:14}
	\end{table}

	\begin{table}[H]
		\centering
		\begin{tabular}{ |p{5cm}|p{10.3cm}| } 
			\hline
			\rowcolor{PineGreen!70}
			\textbf{Nome} & \textbf{Descrizione} \\
			\hline
			Gestione profilo &  Il sistema deve consentire a un utente autenticato di gestire il suo profilo, 
			egli potrà visitare i contenuti inseriti e modificarli. Inoltre potrà cambiare la foto profilo. \\
			\hline
		\end{tabular}
		\caption{RQF.15}
		\label{table:15}
	\end{table}
	

	\begin{table}[H]
		\centering
		\begin{tabular}{ |p{5cm}|p{10.3cm}| }
			\hline
			\rowcolor{PineGreen!70}
			\textbf{Nome} & \textbf{Descrizione} \\
			\hline
			Visualizza compilation & Il sistema deve consentire a un utente autenticato di visualizzare i dettagli
			delle proprie compilation.  \\
			\hline
		\end{tabular}
		\caption{RQF.16}
		\label{table:16}
	\end{table}

	\begin{table}[H]
		\centering
		\begin{tabular}{ |p{5cm}|p{10.3cm}| }
			\hline
			\rowcolor{PineGreen!70}
			\textbf{Nome} & \textbf{Descrizione} \\
			\hline
			Creazione compilation & Il sistema deve consentire a un utente autenticato di creare delle \textit{compilation personalizzate},
			caratterizzate da un titolo e da una descrizione. \\
			\hline
		\end{tabular}
		\caption{RQF.17}
		\label{table:17}
	\end{table}

	\begin{table}[H]
		\centering
		\begin{tabular}{ |p{5cm}|p{10.3cm}| }
			\hline
			\rowcolor{PineGreen!70}
			\textbf{Nome} & \textbf{Descrizione} \\
			\hline
			Elimina compilation & Il sistema deve consentire a un utente autenticato di eliminare le proprie
			compilation personalizzate \\
			\hline
		\end{tabular}
		\caption{RQF.18}
		\label{table:18}
	\end{table}
	
	\begin{table}[H]
		\centering
		\begin{tabular}{ |p{5cm}|p{10.3cm}| }
			\hline
			\rowcolor{PineGreen!70}
			\textbf{Nome} & \textbf{Descrizione} \\
			\hline
			Elimina itinerario da compilation & Il sistema deve consentire a un utente autenticato di eliminare gli itinerari
			che compongono le proprie compilation personalizzate \\
			\hline
		\end{tabular}
		\caption{RQF.19}
		\label{table:19}
	\end{table}

	\begin{table}[H]
		\centering
		\begin{tabular}{ |p{5cm}|p{10.3cm}| }
			\hline
			\rowcolor{PineGreen!70}
			\textbf{Nome} & \textbf{Descrizione} \\
			\hline
			Conversazioni private & Il sistema deve permettere all'utente autenticato lo scambio di informazioni con altri utenti.
			È possibile avviare la conversazione a partire dai contenuti pubblicati. \\
			\hline
		\end{tabular}
		\caption{RQF.20}
		\label{table:20}
	\end{table}

\end{document}	