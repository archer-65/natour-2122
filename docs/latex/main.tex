\documentclass{natourDoc}
\usepackage{lipsum}
\usepackage{tabularx}
\usepackage[table]{xcolor}
\title{Documentazione IngSW} %Titolo

\begin{document}

%----------- Informazioni --------------------

\titolo{NaTour21} %Titolo.pdf
\sottotitolo{Dipartimento di Ingegneria Elettrica \\ e delle Tecnologie dell'Informazione} %Nome progetto

\professors{Sergio \textsc{Di Martino} \\
            Francesco \textsc{Cutugno}} %Nome docenti

\students{Bianca Giada \textsc{Chehade} N86003209 \\
		Mario \textsc{Liguori} N86003258\\ 
		Mattia \textsc{Rossi} N86003211} %Nome studenti

%----------- Inizializzazione -----------------
        
\marginscreate %Crea margini
\cover %Crea cover
\toc %Crea Table of Contents

%------------ Corpo del documento ----------------

% \section{Première section} 

% \lipsum[3-4]%LOREMIPSUM

% \subsection{Subsection}

% \lipsum[3-4] %LOREMIPSUM

% \section{Deuxième section}

% \lipsum[3-5] %LOREMIPSUM

%------------- Comandi utili ----------------

% \section{Quelques commandes}

% Voici quelques commandes utiles :

%------ Per inserire immagini centrate ------

%\insertfigure{logos/logo.png}{3cm}{Légende de la figure}{Label de la figure}
% Le premier argument est le chemin pour la photo
% Le deuxième est la hauteur de la photo
% Le troisième la légende
% Le quatrième le label
%Ici, je cite l'image \ref{fig: Label de la figure}


%------- Per inserire un equazione --------------

% \begin{equation} \label{eq: exemple}
% \rho + \Delta = 42
% \end{equation}

% L'équation \ref{eq: exemple} est cité ici. 

% ------- Per inserire simboli/variabili ----------------------

% Pour écrire des variables dans le texte, il suffit de mettre le symbole \$ entre le texte souhaité comme : constante $\rho$. 

\section{Descrizione del Progetto}
	NaTour21 è un sistema complesso e distribuito finalizzato ad offrire un moderno social network multipiattaforma per appassionati di escursioni.\\

	Il sistema consiste in:
	\begin{itemize}
		\item un Back-End sicuro, performante e scalabile;
		\item un client mobile attraverso cui gli
		utenti possono fruire delle funzionalità del sistema in modo intuitivo, rapido e piacevole;
		\item un client mobile attraverso cui gli amministratori possono gestire i contenuti inseriti in piattaforma.\\
	\end{itemize}

	NaTour21 ha lo scopo di creare una community sicura e affidabile dove condividere la propria passione per l'escursionismo.\\

	In questo scenario, l'utente si configura come protagonista: oltre alla possibilità di inserire itinerari (dettagliati da informazioni), compilation e post personali sulla piattaforma,
	è lasciato ampio spazio all'individualità personale.\\

	Tutto ciò si concretizza con la possibilità di interagire con gli altri utenti, in modo da poter avere un contatto più diretto con
	la realtà dell'escursionismo, e di lasciare valutazioni personali su qualunque itinerario si desideri.\\

	Il sistema valuta essenziale la sicurezza degli utenti: questi potranno segnalare informazioni inesatte o contenuti inappropriati, al fine 
	di rendere la permanenza nella community piacevole.

\section{Documento dei Requisiti Software}
	\subsection{Requisiti funzionali}
	In questa sezione saranno esposti i requisiti funzionali dell'applicazione NaTour21, cioè le funzionalità richieste dai commissionanti.

	\begin{center}
		\begin{tabular}{ |p{5cm}|p{10.3cm}| } 
		 \hline
		 \rowcolor{gray!30}
		 \textbf{Nome} & \textbf{Descrizione} \\
		 \hline
		  Registrazione & Il sistema deve consentire ad un utente di potersi registrare indicando e-mail e password, oppure utilizzando account di terze parti (Google o Facebook).\\ 
		 \hline
		\end{tabular}
		\end{center}

\end{document}
