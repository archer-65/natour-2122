\documentclass{natourDoc}
\usepackage{lipsum}
\title{Documentazione IngSW} %Titolo

\begin{document}

%----------- Informazioni --------------------

\titolo{NaTour21} %Titolo.pdf
\sottotitolo{Dipartimento di Ingegneria Elettrica \\ e delle Tecnologie dell'Informazione} %Nome progetto

\professors{Sergio \textsc{Di Martino} \\
            Francesco \textsc{Cutugno}} %Nom de l'enseignant

\students{Bianca Giada \textsc{Chehade} N86003209 \\
		Mario \textsc{Liguori} N86003258\\ 
		Mattia \textsc{Rossi} N86003211} %Nom des élèves

%----------- Inizializzazione -----------------
        
\marginscreate %Crea margini
\cover %Crea cover
\toc %Crea Table of Contents

%------------ Corpo del documento ----------------

% \section{Première section} 

% \lipsum[3-4]%LOREMIPSUM

% \subsection{Subsection}

% \lipsum[3-4] %LOREMIPSUM

% \section{Deuxième section}

% \lipsum[3-5] %LOREMIPSUM

%------------- Comandi utili ----------------

% \section{Quelques commandes}

% Voici quelques commandes utiles :

%------ Per inserire immagini centrate ------

%\insertfigure{logos/logo.png}{3cm}{Légende de la figure}{Label de la figure}
% Le premier argument est le chemin pour la photo
% Le deuxième est la hauteur de la photo
% Le troisième la légende
% Le quatrième le label
%Ici, je cite l'image \ref{fig: Label de la figure}


%------- Per inserire un equazione --------------

% \begin{equation} \label{eq: exemple}
% \rho + \Delta = 42
% \end{equation}

% L'équation \ref{eq: exemple} est cité ici. 

% ------- Per inserire simboli/variabili ----------------------

% Pour écrire des variables dans le texte, il suffit de mettre le symbole \$ entre le texte souhaité comme : constante $\rho$. 


\end{document}
